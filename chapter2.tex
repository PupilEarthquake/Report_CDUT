\chapter{各种测试各种测试各种测试各种测试各种测试各种测试各种测试}

{\setlength{\parindent}{0em}
Radiohead are an English rock band formed in Abingdon, Oxfordshire, in 1985. The band consists of Thom Yorke (vocals, guitar, piano, keyboards); brothers Jonny Greenwood (lead guitar, keyboards, other instruments) and Colin Greenwood (bass); Ed O'Brien (guitar, backing vocals); and Philip Selway (drums, percussion). They have worked with the producer Nigel Godrich and the cover artist Stanley Donwood since 1994. Radiohead's experimental approach is credited with advancing the sound of alternative rock.

Radiohead signed to EMI in 1991 and released their debut album, Pablo Honey, in 1993. Their debut single, "Creep", became a worldwide hit. Radiohead's popularity and critical standing rose with the release of The Bends in 1995. Radiohead's third album, OK Computer (1997), brought them international fame; noted for its complex production and themes of modern alienation, it is acclaimed as a landmark record and one of the best albums in popular music.

Radiohead's fourth album, Kid A (2000), marked a dramatic change in style, incorporating influences from electronic music, jazz, classical music and krautrock. Though Kid A divided listeners, it was later named the best album of the decade by multiple outlets. It was followed by Amnesiac (2001), recorded in the same sessions. Hail to the Thief (2003), with lyrics addressing the War on Terror, blended the band's rock and electronic sides, and was Radiohead's final album for EMI.

Radiohead self-released their seventh album, In Rainbows (2007), as a download for which customers could set their own price, to critical and chart success. Their eighth album, The King of Limbs (2011), an exploration of rhythm, was developed using extensive looping and sampling. A Moon Shaped Pool (2016) prominently featured Jonny Greenwood's orchestral arrangements. Yorke, Jonny Greenwood, Selway, and O'Brien have released solo albums; in 2021, Yorke and Jonny Greenwood debuted a new band, the Smile.

By 2011, Radiohead had sold more than 30 million albums worldwide. Their awards include six Grammy Awards and four Ivor Novello Awards, and they hold five Mercury Prize nominations, the most of any act. Seven Radiohead singles have reached the top 10 on the UK Singles Chart: "Creep" (1992), "Street Spirit (Fade Out)" (1996), "Paranoid Android" (1997), "Karma Police" (1997), "No Surprises" (1998), "Pyramid Song" (2001), and "There There" (2003). "Creep" and "Nude" (2008) reached the top 40 on the US Billboard Hot 100. Rolling Stone named Radiohead one of the 100 greatest artists of all time, and Rolling Stone readers voted them the second-best artist of the 2000s. Five Radiohead albums have been included in Rolling Stone's 500 Greatest Albums of All Time lists. Radiohead were inducted into the Rock and Roll Hall of Fame in 2019.
}

\section{为什么$e^{i\pi} + 1 = 0$}
一项举动不像年轻人想的那样,有如捡起来丢出去的一颗石头,要不是打中目标,就是错过目标,然后就完毕了。一颗石头被捡起来,土地因而变轻,拿石头的手因而变重。把石头丢出去时,天上星辰以绕行相应。石头打中或坠落,宇宙都因之改变。整体的均衡,仰赖每项单一行动。风、海、水、地与光的力量,以及禽兽植物都如此,一切都完好、合宜地搭配着。这一切行动都包括在“一体至衡”当中。举凡飓风、大鲸鱼的号鸣、枯叶的吹落、蚊蚋的飞移,一切行动都在整体均衡的范围内。我们,既然身为具备力量操控世界、并相互操控的人,就必须学会按照落叶、鲸鱼、风的本性去行动。我们必须学会保持那均衡。


世上没有安全,没有尽头。人必须在寂静中,才能听见世界的声音。必须在黑暗中,才能看见星星。若要跳舞,永远要在虚空处,要在恐怖的深渊之上,才算舞蹈。


在我们心里,孩子,在我们心里。我们内心那个叛徒、那个自我,那个哭喊着“我要活下去,只要我能活活下去,让人间任意败坏去吧!”的自我,我们内在那个背逆的灵魂,躲在黑暗中,犹如关在箱里的蜘蛛。他对我们大家说话,但只有少数人听懂,不外乎巫师、歌者、制造者与英雄豪杰这些努力要成为自己的人。“成为自己”是稀罕的事,也是了不起的事,永远当“自己”,岂非更了不起?\cite{LeGuin}

