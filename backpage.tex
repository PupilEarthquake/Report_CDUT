\begin{table}[h]
    \begin{tabularx}{\textwidth}{|p{2em}|X|}
        \hline
        \heiti{学生实验心得}&\qquad 陈献章字公甫,新会之白沙里人。身长八尺,目光如星,右脸有七黑子,如北斗状。自幼警悟绝人,读书一览辄记。尝读《孟子》所谓天民者,慨然曰:“为人必当如此!”梦拊石琴,其音泠泠然,一人谓之曰:“八音中惟石难谐,子能谐此,异日其得道乎?”因别号石斋。正统十二年举广东乡试,明年会试中乙榜,入国子监读书。已至崇仁,受学於康斋先生,归即绝意科举,筑春阳台,静坐其中,不出阈外者数年。寻遭家难。成化二年,复游太学,祭酒邢让试和杨龟山《此日不再得》诗,见先生之作,惊曰:“即龟山不如也。”扬言於朝,以为真儒复出,由是名动京师。罗一峰、章枫山、庄定山、贺医闾皆恨相见之晚,医闾且禀学焉。归而门人益进。十八年,布政使彭韶、都御史朱英交荐,言“国以仁贤为宝,臣自度才德不及献章万万,臣冒高位,而令献章老丘壑,恐坐失社稷之宝”。召至京,阁大臣或尼之,令就试吏部。辞疾不赴,疏乞终养,授翰林院检讨而归。有言其出处与康斋异者,先生曰:“先师为石亨所荐,所以不受职;某以听选监生,始终愿仕,故不敢伪辞以钓虚誉,或受或不受,各有攸宜。”自后屡荐不起。弘治十三年二月十日卒,年七十有三。先生疾革,知县左某以医来,门人进曰:“疾不可为也。”先生曰:“须尽朋友之情。”饮一匙而遣之。\\
        &\\
        &\\
        &\\
        &\\
        &\hspace{9cm} 学生(签名): \\
        & \\
        &\hfill 202X 年 XX 月 XX 日 \\
        \hline
        \heiti{诚信承诺}& \qquad 本人郑重声明所呈交的实习报告是本人在指导教师指导下进行的研究工作及取得的研究成果。据我所知,除了文中特别加以标注的地方外,报告中不包含其他人已经发表或撰写过的研究成果。与我一同工作的同学对本文研究所做的贡献均已在报告中作了明确的说明并表示谢意。\\
        &\\
        &\\
         &\hspace{9cm} 学生(签名): \\
        & \\
        \hline
        \heiti{指导老师评语}& \\
        &\\
        &\\
        &\\
        &\\
        &\\
        &\\
        &\\
        &\\
        &\hspace{9cm} 成绩评定:\\
        &\\
        &\hspace{9cm} 指导老师签名:\\
        &\\
        &\hfill 年\qquad 月\qquad 日 \\
    \hline
    \end{tabularx}
\end{table}