\ProvidesFile{Font_set.tex}[2023]

% 中文采用Fondol系列字体
% Fandol系列字体仓库地址: https://github.com/Yixf-Self/fandol-font

\setmainfont{Times New Roman} % 设置西文主字体
\setsansfont{TeX Gyre Heros} % 设置西文无衬线字体

\setCJKmainfont[
    BoldFont=FandolSong-Bold.otf,
    ItalicFont=FandolKai-Regular.otf
    ]{FandolSong-Regular.otf}

\setCJKsansfont[
    BoldFont=FandolHei-Bold.otf
    ]{FandolHei-Regular.otf}

\setCJKmonofont[
    BoldFont=FandolHei-Bold.otf,
    ]{FandolHei-Regular.otf}

\setCJKfamilyfont{song}[
    BoldFont=FandolSong-Bold.otf,
    ItalicFont=FandolKai-Regular.otf
    ]{FandolSong-Regular.otf}

\setCJKfamilyfont{kai}[
    BoldFont=FandolKai-Regular.otf,
    ItalicFont=FandolKai-Regular.otf
]{FandolKai-Regular.otf}

\setCJKfamilyfont{hei}[
    BoldFont=FandolHei-Bold.otf,
    ItalicFont=FandolHei-Regular.otf
]{FandolHei-Regular.otf}


\newcommand\kaishu{\CJKfamily{kai}} % 该命令对后面所有文本起作用
\newcommand\songti{\CJKfamily{song}}
\newcommand\heiti{\CJKfamily{hei}}